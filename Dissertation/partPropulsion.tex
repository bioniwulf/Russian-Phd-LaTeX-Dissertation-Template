\chapter{Аналитическое описание ДРК и динамических моделей его составных элементов}\label{ch:Propulsion}

Исполнительные элементы ДРК определяют эффективность траекторного маневрирования НПА, а также возможность его динамического позиционирования в точке или зависания в толще воды.
На практике используются различные конструктивные схемы ДРК, в состав которых могут входить маршевые и подруливающие движители, носовые и кормовые рулевые устройства.

\section{Формальное описание структуры ДРК}\label{sec:Propulsion/System}
Движители в системе можно разделить на три больших класса \cite{армишев86}:
\begin{itemize}
    \item \textbf{фиксированные}, когда вектор упора может изменяться только вдоль фиксированной относительно ПА прямой.
    \item \textbf{поворотные} (азимутальные, azimuth), когда вектор упора может изменяться в фиксированной относительно ПА плоскости.
    Сюда относятся гребные винты на поворотных колонках, гребные винты с поворотными насадками, крыльчатые движители и т.д.
    \item \textbf{пространственные}, когда вектор упора может изменяться в пространстве.
\end{itemize}
Следует отметить, что если первые два класса движителей были заимствованы для ПА с надводных судов, то пространственные движители были разрабтаны для решения специфических задач подводных аппаратов.

Определим формально, как по структуре ДРК находится число степеней свободы $m$, которые можно контролировать движение данного ПА, и наоборот, какова должна быть структура ДРК, что бы можно было управлять по заданным степеням свободы.

Пусть $Oxyz$ -- правая ортогональная система координат жестко связанная с ПА. Ось $Ox$ направлена из кормы в нос, ось $Oy$ с левого борта на правый, а ось $Oz$ дополняет систему до правой ортогональной (SNAME нотация системы координат).

Введём следующие обозначения:
\begin{itemize}
    \item $n_f \geq 0, n_a \geq 0, n_r \geq 0$ -- число, соответственно, фиксированных, поворотных и пространственных движителей входящих в состав ДРК;
    % \item $i$ -- номер по порядку каждого движителя, где $i$ меняется в диапазоне $1,2, \ldots, n_f+n_a+n_r$;
    \item $u^{i}, i < n_f$ - величина упора фиксированного движителя;
    \item $(u^{i}_x, u^{i}_y, u^{i}_z)$, где $i$ меняется в диапазоне $n_f+1, n_f+2, \ldots, n_f+n_a+n_r$ -- проекции векторов упоров поворотных и пространственных движителей, соответственно на оси $Ox, Oy, Oz$;
    \item $(P^i_x, P^i_y,P^i_z)$ -- координаты точки крепления движителя с номером $i$ относительно $Oxyz$;
    \item $(C^i_x, C^i_y, C^i_z)$ -- направляющие косинусы единичного вектора, направленного вдоль линии действия фиксированного движителя с номером $i$;
    \item $(R^i_x, R^i_y, R^i_z)$, где $i$ меняется в диапазоне $n_f+1, n_f+2, \ldots, n_f+n_a$ -- направляющие косинусы вектора единичной нормали к плоскости вращения поворотного движителя с номером $i$;
    \item $(\nu_x, \nu_y, \nu_z)$ -- проекции главного вектора управляющих сил на оси $Ox, Oy, Oz$;
    \item $(\nu_{mx},\nu_{my},\nu_{mz})$ -- проекции главного вектора момента управляющих сил на оси $Ox, Oy, Oz$.
    \item $\vect{u}=[u^{1}, u^{2}, \ldots, u^{n_f+n_a+n_r}_x, u^{n_f+n_a+n_r}_y, u^{n_f+n_a+n_r}_z]^T$ -- обобщенный вектор упоров создаваемый ДРК;
    \item $\vect{\nu} = [\nu_x, \nu_y, \nu_z, \nu_{mx},\nu_{my},\nu_{mz}]^T$ -- обобщенный вектор сил и моментов действующий на аппарат в системе координат $Oxyz$.
\end{itemize}

Можно показать что между векторами $\vect{u}$  и $\vect{\nu}$ есть линейная зависимость, которая выражается через матрицу $B$:
\begin{equation}
    \label{eq:propulsion_connection}
    B\vect{u} = \vect{\nu}
\end{equation}

В \ref{eq:propulsion_connection} шесть уравнений определяют связь между проекциями сил и моментов в связанной с аппаратом системе координат $Oxyz$ с одной стороны и $n_f$ упоров фиксированных и $3(n_r+n_r)$ проекций упоров поворотных и пространственных движителей с другой стороны.

Однако не все создаваемые движителями упоры являются независимыми.
Вектор тяги поворотного движителя должен всегда лежать в плоскости вращения. 
Для этого вводят расширенный вектор сил и моментов $\vect{\nu^e} = [\vect{\nu}, 0, \ldots, 0]$, где $\vect{\nu^e} \in \mathspace{R}^{6+n_a}$ и соответствующую матрицу $B^e$:
\begin{equation}
    \label{eq:propulsion_connection_enchanced}
    B^e\vect{u} = \vect{\nu^e}
\end{equation}
где последние $n_a$ уравнений отражают это ограничение накладываемое на вращательные движители.

Общие вид матрицы $B^e=(B^e_f, B^e_a, B^e_r)$, где $B^e_f \in \mathspace{R}^{(6 + n_f) \times n_f}$ (\ref{eq:propulsion_matrix_fix}), $B^e_a \in \mathspace{R}^{(6 + n_f) \times 3n_a}$ (\ref{eq:propulsion_matrix_azimuth}), $B^e_r \in \mathspace{R}^{(6 + n_f) \times 3n_r}$ (\ref{eq:propulsion_matrix_rotation}) -- матрицы отражающая влияние фиксированных движителей, поворотных и пространственных движителей соответственно.

\begin{equation}
    \label{eq:propulsion_matrix_fix}
    B^e_f = 
    \begin{pmatrix}
        C^1_x & \ldots & C^{n_f}_x \\
        C^1_y & \ldots & C^{n_f}_y \\
        C^1_z & \ldots & C^{n_f}_z \\
        [C^1 \times P^1]_x & \ldots & [C^{n_f} \times P^{n_f}]_x \\
        [C^1 \times P^1]_y & \ldots & [C^{n_f} \times P^{n_f}]_y \\
        [C^1 \times P^1]_z & \ldots & [C^{n_f} \times P^{n_f}]_z \\
        \ldots & \ldots & \ldots \\
        0 & \ldots & 0 \\
    \end{pmatrix}
\end{equation}

\begin{equation}
    \label{eq:propulsion_matrix_azimuth}
    B^e_a = 
    \begin{pmatrix}
        1 & 0 & 0 & \ldots & 1 & 0 & 0 \\
        0 & 1 & 0 & \ldots & 0 & 1 & 0 \\
        0 & 0 & 1 & \ldots & 0 & 0 & 1 \\
        0 & -P_z^1 & P_y^1 & \ldots & 0 & -P_z^{n_a} & P_y^{n_a} \\
        P_z^1 & 0 & -P_x^1 & \ldots & P_z^{n_a} & 0 & -P_x^{n_a} \\
        -P_y^1 & P_x^1 & 0 & \ldots & -P_y^{n_a} & P_x^{n_a} & 0 \\
        R_x^1 & R_y^1 & R_z^1 & \ldots & 0 & 0 & 0 \\
        \ldots & \ldots & \ldots & \ldots & \ldots & \ldots & \ldots \\
        0 & 0 & 0 & \ldots & R_x^{n_a} & R_y^{n_a} & R_z^{n_a} \\
    \end{pmatrix}
\end{equation}

\begin{equation}
    \label{eq:propulsion_matrix_rotation}
    B^e_r = 
    \begin{pmatrix}
        1 & 0 & 0 & \ldots & 1 & 0 & 0 \\
        0 & 1 & 0 & \ldots & 0 & 1 & 0 \\
        0 & 0 & 1 & \ldots & 0 & 0 & 1 \\
        0 & -P_z^1 & P_y^1 & \ldots & 0 & -P_z^{n_a} & P_y^{n_a} \\
        P_z^1 & 0 & -P_x^1 & \ldots & P_z^{n_a} & 0 & -P_x^{n_a} \\
        -P_y^1 & P_x^1 & 0 & \ldots & -P_y^{n_a} & P_x^{n_a} & 0 \\
        R_x^1 & R_y^1 & R_z^1 & \ldots & 0 & 0 & 0 \\
        \ldots & \ldots & \ldots & \ldots & \ldots & \ldots & \ldots \\
        0 & 0 & 0 & \ldots & 0 & 0 & 0 \\
    \end{pmatrix}
\end{equation}

Матрица $B^e$ зависит от расположения и ориентации движителей, а её размеры от числа различных типов движителей.
Следовательно она зависит только от структуры ДРК, в связи с чем будем её называть -- \textbf{матрицей структуры движительно-рулевого комплекса подводного аппарата}.
Формально многие вопросы, связанные с выбором структуры ДРК, могут быть решены с помощью матрицы структуры ДРК.

Элементы матрицы зависят от выбора системы координат.
Легко проверить, что в общем случае ортогонального преобразования (поворот и параллельный перенос) перевода в новую систему координат, когда произвольный вектор $\vect{v}=[v_x,v_y,v_z]^T$ переходит в $\vect{q}=[q_x,q_y,q_z]^T$:
\begin{equation*}
    \left\{
    \begin{array}{ll}
         &\vect{q} = R\vect{v} + \vect{d}\\
         &\det[R] = 1
    \end{array}
    \right.
\end{equation*}
\noindent где $R$ -- матрица поворота, $\vect{d}$ -- вектор линейного переноса в систему координат $(Oxyz)')$.

Матрица $B^e$ в новой системе координат будет задана следующим образом:
\begin{equation}
    B^e_{(Oxyz)'} = Q_2Q_3B^eQ_1
\end{equation}
\noindent где $B^e_{(Oxyz)'}$ -- матрица конфигурации ДРК в новой системе координат, $Q_1 \in \mathspace{R}^{(n_f+3n_a+3n_r)\times(n_f+3n_a+3n_r)}$ (\ref{eq:matrix_rotation_q1}), $Q_2 \in \mathspace{R}^{(6+n_a)\times(6+n_a)}$ (\ref{eq:matrix_rotation_q2}), $Q_3 \in \mathspace{R}^{(6+n_a)\times(6+n_a)}$ (\ref{eq:matrix_rotation_q3}) -- диагональные матрицы перехода в $(Oxyz)'$.

\begin{equation}
    \label{eq:matrix_rotation_q1}
    Q_1 = 
    \setlength{\arraycolsep}{0pt}
    \begin{pNiceMatrix}[columns-width=auto]
        I^{n_f\times n_f} &     &         & \text{\Large0} \\
                          & R^T &         &                \\
                          &     & \ddots  &                \\
        \text{\Large0}    &     &         & R^T
    \end{pNiceMatrix}
\end{equation}
\noindent где:
\begin{itemize}
    \item $I^{n_f\times n_f} \in \mathspace{R}^{n_f \times n_f}$ -- единичная диагональная матрица;
\end{itemize}

\begin{equation}
    \label{eq:matrix_rotation_q2}
    Q_2 = 
    \setlength{\arraycolsep}{0pt}
    \begin{pNiceMatrix}[columns-width=auto]
        R               &     &         & \text{\Large0} \\
                        & R   &         &                \\
                        &     & \ddots  &                \\
        \text{\Large0}  &     &         & I^{n_a\times n_a}
    \end{pNiceMatrix}
\end{equation}

\begin{equation}
    \label{eq:matrix_rotation_q3}
    Q_3 = 
    \setlength{\arraycolsep}{0pt}
    \begin{pNiceMatrix}[columns-width=auto]
        W & 0 \\
        0 & I^{n_a \times n_a}
    \end{pNiceMatrix}
\end{equation}

\noindent где матрица $W$ определяется следующим выражением:
\begin{equation*}
    \setlength{\arraycolsep}{0pt}
    \begin{pNiceMatrix}[columns-width=auto]
        1    & 0    & 0    & 0 & 0 & 0 \\
        0    & 1    & 0    & 0 & 0 & 0 \\
        0    & 0    & 1    & 0 & 0 & 0 \\
        0    & -d_z & d_y  & 1 & 0 & 0 \\
        d_z  & 0    & -d_x & 0 & 1 & 0 \\
        -d_y & d_x  & 0    & 0 & 0 & 1
    \end{pNiceMatrix}
\end{equation*}

По виду матриц $Q_1, Q_2, Q_3$ следует что их определители совпадают и равны единице:
\begin{equation*}
    \det Q_1 = \det Q_2 = \det Q_3 = 1
\end{equation*}

Отсюда следует что при ортогональном преобразовании ранг матрицы ДРК $B^e$ не изменяется:
\begin{equation*}
    \text{rg}B^e = \text{rg}B^e_{(Oxyz)'} = \text{const}
\end{equation*}

Рассмотрим как связаны свойства матрицы структуры движительно-рулевого комплекса ПА с возможностью управления по заданным степеням свободы.
Это означает, что данный ДРК может одновременно создавать заданные вектора управляющей силы и момента или только некоторых из их проекций.
Так, например, для управления по четырем степеням свободы (продольное и вертикальное перемещение, курс, дифферент), необходимо что бы ДРК создавал две заданные проекции управляющей силы и две проекции момента.
Необходимое число контролируемых степеней свободы существенно зависит от целевых задач ПА. 

Для необитаемых ПА необходимо создавать достаточно сложные ДРК, способные управлять по 5-6 степеням свободы.
На обитаемых ПА, как правило, управляют по четырем степеням свободы (вперед, вверх, лаг, курс).
Для управления же по крену и дифференту на обитаемых ПА чаще используются перемещаемые грузы или жидкости -- крен-дифферентные системы.
Однако, и на обитаемых ПА в ряде случаев требуются высокоточное управление по всем шести степеням свободы.

Рассмотрим наиболее сложные структуры ДРК, предназначенные для управления по шести степеням свободы ($m=6$).
Ранг матрицы структуры ДРК должен удовлетворять условию:
\begin{equation}
    \label{eq:propulsion_matrix_rank}
    \text{rg}B^e=6+n_a
\end{equation}
Причём данное условие является необходимым и достаточным.
То есть, если данный ДРК способен управлять одновременно по $m=6$ степеням свободы, то его матрица структуры удовлетворяет данному условию и наоборот.

В общем случае для $m\leq 6$ аналогичное условие будет записано следующим образом:
\begin{equation}
    \label{eq:propulsion_matrix_rank_com}
    \text{rg}B^e \geq m + n_a
\end{equation}

Пусть $m=6$ и требуется создать вектор сил и моментов $\vect{\nu}$.
Для решения задачи распределения вектора упоров между движителями необходимо решить систему линейных уравнений \ref{eq:propulsion_connection}.
При этом можно показать что при выполнении условия \ref{eq:propulsion_matrix_rank_com} решение системы \ref{eq:propulsion_connection} всегда существует, однако может быть не единственным.
Это означает что заданную управляющую силу можно создать при различных сочетаниях векторов упоров на движителях.
Такие ДРК называются \textbf{избыточными}, в отличие от неизбыточных, когда распределение упоров можду движителями может быть единственным.
избыточные ДРК могут применяться и для управляения по меньшему числу степеней свободы $m \leq 6$.

Формально вопрос избыточности ДРК решаетя по виду матрицы $B^e$.
Как нетрудно установить, ДРК будет неизбыточным для управления по $m=6$ степеням свободы, когда матрица $B^e$ квадратная и выполняется условие \ref{eq:propulsion_matrix_rank}, то есть:
\begin{equation}
    \left\{
    \begin{array}{ll}
    n_f + 2n_a + 3n_r = 6  \\
    \det B^e \neq 0 
    \end{array}
    \right.
\end{equation}
\noindent и избыточным в случае:
\begin{equation}
    \left\{
    \begin{array}{ll}
    n_f + 2n_a + 3n_r > 6  \\
    \text{rg}B^e = 6 + n_a
    \end{array}
    \right.
\end{equation}

В общем случае для $m \leq 6$ неизбыточные ДРК определяются следующим образом:
\begin{equation}
    \left\{
    \begin{array}{ll}
    n_f + 2n_a + 3n_r = N  \\
    \text{rg}B^e = N + n_a
    \end{array}
    \right.
\end{equation}
\noindent а избыточные:
\begin{equation}
    \left\{
    \begin{array}{ll}
    n_f + 2n_a + 3n_r > \text{rg}B^e  \\
    \text{rg}B^e \geq N + n_a
    \end{array}
    \right.
\end{equation}

\section{Учет рулей управления в матрице структуры ДРК ПА}
Уравнение \ref{eq:propulsion_connection} не поддерживает рули управления как один из способов контроля движения ПА.
Это исторически отдельная задача и ей посвящено достаточно много литературы в области управления летательными аппаратами, но рули управления также широко распространены и в подводных аппаратах, и задача распределения управляющих воздействий между этими исполнительными механизмами остаётся актуальной.

Так например в работе \cite{10.1177/1729881417741738} закон линейной взаимосвязи между обобщенным вектором силы и момента в связанной системе координат BODY $\vect{\nu}$ и вектором углов поворота движителей $\vect{\delta}$, определяется следующим образом:
\begin{equation}
    \begin{pmatrix}
        X \\
        Y \\
        Z \\
        K \\
        M \\
        N
    \end{pmatrix}
    = u^2
    \begin{pmatrix}
        X_{\delta_1\delta_1} & X_{\delta_2\delta_2} & X_{\delta_3\delta_3} & X_{\delta_4\delta_4} \\
        Y_{\delta_1} & Y_{\delta_2} & Y_{\delta_3} & Y_{\delta_4} \\
        Z_{\delta_1} & Z_{\delta_2} & Z_{\delta_3} & Z_{\delta_4} \\
        K_{\delta_1} & K_{\delta_2} & K_{\delta_3} & K_{\delta_4} \\
        M_{\delta_1} & M_{\delta_2} & M_{\delta_3} & M_{\delta_4} \\
        N_{\delta_1} & N_{\delta_2} & N_{\delta_3} & N_{\delta_4}
    \end{pmatrix}
    \begin{pmatrix}
        \delta_1 \\
        \delta_2 \\
        \delta_3 \\
        \delta_4
    \end{pmatrix}
\end{equation}
\noindent
\begin{itemize}
    \item $\nu = [X, Y, Z, K, M, N]^T$ -- обобщенный вектор сил вдоль продольной, поперечно и нормальной осью ССК, и моментов вокруг них;
    \item $\vect{\delta} = [\delta^1, \delta^2, \delta^3, \delta^4]^T$ -- вектор углов поворота рулей, где $\delta^1, \delta^2, \delta^3, \delta^4$ -- углы поворота соответственно верхнего левого, верхнего правого, нижнего левого и нижнего правого кормового руля управления;
    \item $u$ -- скорость потока воды набегаемой на аппарат при его движении;
    \item $X_{\delta_i, \delta_i}, \ldots, N_{\delta_i}$ -- коэффициент сил и моментов создаваемых РУ при скорости набегаемого потока $u$.
\end{itemize}

\section{Модели динамики исполнительных элементов}\label{sec:Propulsion/Models}
Наиболее распространенными типами движителей являются гребные винты в насадке и водометные движители, которые могут устанавливаться стационарно на корпусе аппарата или на поворотных кронштейнах, которые поворачиваются на требуемый угол в плоскости или пространстве для изменения направления действия силы тяги.
При этом использование водометных движителей ограничено их сравнительно низким КПД (0.5–0.55) по сравнению с гребными винтами, у которых он может достигать значений 0.7–0.75 \cite{инзарцев2018подводные}. Гораздо меньший КПД имеют такие экзотические движительные установки, как крыльчатые, волновые или машущие. 

Рулевые устройства, использующие гидродинамические крылья в качестве исполнительного органа, как известно, имеют низкую эффективность при малых скоростях набегающего потока \cite{агеев2015авто}.
При этом на крейсерских скоростях движения использование носовых и кормовых рулей направления и глубины имеет очевидное преимущество по сравнению с подруливающими движителями в части энергопотребления.
Остановимся на традиционных исполнительных элементах ДРК многофункционального НПА, обеспечивающего выполнение обзорно-поисковых работ с движением в широком диапазоне скоростей хода и динамическое позиционирование в толще воды.

Для корректного и энергетически эффективного решения задачи распределения управляющего воздействия на исполнительные механизмы движительно-рулевого комплекса необходимо корректно статически описать как сам комплекс исполнительных механизмов, так и динамические процессы происходящие в них с учётом гидродинамических особенностей их поведения.

\subsection{Модель маршевого движителя}
Это один или несколько кормовых движителей, обеспечивающих продольное движение аппарата, а также возможность маневрирования по глубине и курсу.

В работе \cite{10.1109/48.107145} была предложена модель движителя заданная в форме пространства состояний по входному параметру $n$ где $n$ -- скорость вращения вала:
\begin{gather}
    \label{eq:thruster_dynamic_1}
    J_m\dot{n} + K_{n|n|}n|n| = Q \\
    T = T(n, u_p)
\end{gather}

\noindent где:
\begin{itemize}
    \item $Q$ -- момент сопротивления вращению на валу привода;
    \item $J_m$ -- момент инерции на привода/пропеллера (кг$\cdot$м$^2$);
    \item $K_{n|n|}$ -- нелинейный коэффициент демпфирования мотора (кг$\cdot$м$^2$);
    \item $u_p$ -- скорость потока в канале движителя;
\end{itemize}

Коэффициент $K_{n|n|}$ определяется следующим образом:
\begin{gather}
    K_{n|n|} = \frac{1}{2}\eta^3 \cdot p^3 \cdot \rho \cdot A \\
    J_m = \eta^2 \cdot p^2 \cdot \rho \cdot V
\end{gather}
где:
\begin{itemize}
    \item $\eta$ -- безразмерный коэффициент эффективности винта;
    \item $p$ -- шаг винта (продольное расстояние которое винт за один оборот), (pitch);
    \item $A$ -- площадь сечения винта, (duct area);
    \item $V$ -- объем насадки в которой расположен винт.
\end{itemize}

Позже модель, представленная уравнением \ref{eq:thruster_dynamic_1}, была усовершенствована в работе \cite{10.1109/48.468242} на основе исследований \cite{cody1992experimental, mclean1991dynamic}.
Новая модель описывает вектором из двух состояний и записывается следующим образом:
\begin{gather}
    J_m\dot{n} + K_{n|n|}n|n| = Q_m- Q \\
    m_f \dot{u_p} + d_f(u_p-u) |u_p - u| = T \\
    T = T(n,u_p) \\
    Q = Q(n,u_p)
\end{gather}
\noindent где:
\begin{itemize}
    \item $Q_m$ -- момент на валу сформированный приводом (Нм);
    \item $m_f$ -- масса воды захватываемая винтом (кг);
    \item $d_f$ -- квадратичный демпфирующий коэффициент (кг/м);
    \item $u$ -- продольная скорость движения ПА.
\end{itemize}

Экспериментальная проверка обоих подходов была проведена в работе \cite{whitcomb1999development}.

Кроме того, в работе \cite{blanke2000dynamic} рассматривается трехэтапная модель винта, которая уже учитывает упрощённую модель ПА для более точного расчета скорости потока набегающего на грибной винт:
\begin{gather}
    J_m\dot{n} + K_{n|n|}n|n| = \tau - Q \\
    m_f \dot{u_p} + d_{f0}u_p + d_f|u_p|(u_p - u_a) = T \\
    (m - X_{\dot{u}})\dot{u} - X_u u - X_{u|u|}u|u| = (1-t)T\\
    T = T(n,n_p) \\
    Q = Q(n,n_p)
\end{gather}

Последние исследования в области описания динамики движителя представлены в работе \cite{10.1109/robot.2005.1570115}.

Кроме того, разумно рассмотреть декомпозицию задачи и независимо рассмотреть стационарнуб модель гребного винта и модель динамики электрического привода.

\paragraph{Модель гребного винта}

Для гребного винта с фиксированным шагом, момент на валу $Q$ и сила (упор) вырабатываемая винтом $T$ зависит от продольной скорости движения аппарата $u$, скорости потока воды $u_a$, а так же скорости вращения винта $n$.
Кроме этого на работу движителя влияют нестационарные потоки и различные эффекты которые влияют на эффективность работы движителя.
В соответствии с работами \cite{newman2018marine, breslin1996hydrodynamics, carlton2018marine} к особо влияющим эффектам можно отнеси следующие:
\begin{itemize}
    \item всасывание воздуха;
    \item кавитация;
    \item эффект Вагнера при вращении винта у раздела сред;
    \item эффекты связанные с влиянием морских волн;
    \item эффект Куснера (срыв потока).
\end{itemize}

Приведенные факторы рассматриваются в основном в приложении к суднам.
Для движителей подводных аппаратов, которые большую часть времени погружены под воду, первыми тремя эффектами можно пренебречь.
Эффект Куснера связанный со срывом потока связан с высокоамплитудными переключениями сигнала управления движителем и им можно пренебречь.

Квазистационарное моделирование тяги и крутящего момента движителя обычно происходит в терминах кривых подъемной силы и момент сопротивления, которые возможно преобразовать в тягу и момент.
Подъемная сила и момент сопротивления обычно формируются через безразмерные коэффициенты упора $K_T$ и момента $K_Q$, которые рассчитываются на базе швартовых испытаний движителя.
В ходе этих испытаний формируются кривые зависимости упора и момента от скорости вращения вала движителя
\begin{gather}
\label{eq:propeler_model}
    K_T (J_0) = \frac{T}{\rho D^4 n |n|}, \:
    K_Q (J_0) = \frac{Q}{\rho D^5 n |n|}
\end{gather}
\noindent где $D$ -- диаметр винта, $\rho$ -- плотность воды, а $J_0$ представляет собой относительную поступь и определяется следующим уравнением:
\begin{equation}
    J_0 = \frac{u_a}{nD}
\end{equation}
\noindent где $u_a$ -- скорость окружающей воды (совпадает со скоростью движения аппарата в случае установившегося движения).

Численные расчеты коэффициентов $K_T$, $K_Q$ обычно проводят на натурных экспериментах в специальных бассейнах.
При этом, обычно, нестационарными эффектами пренебрегают.

В свою очередь, сами безразмерные коэффициенты $K_T$, $K_Q$ могут быть описаны как функции со следующими параметрами (\cite{oosterveld1975further}):
\begin{gather}
    K_T = f_1 \left( J_0, \frac{P}{D}, \frac{A_E}{A_0}, Z \right) \\
    K_Q = f_2 \left( J_0, \frac{P}{D}, \frac{A_E}{A_0}, Z, R_n, \frac{t}{c} \right) 
\end{gather}
\noindent где:
\begin{itemize}
    \item $P/D$ -- шаговое отношение винта;
    \item $A_E/A_0$ -- отношение площади винта к площади движителя;
    \item $Z$ -- количество лопастей винта;
    \item $R_n$ -- число Рейнольдса;
    \item $t$ -- максимальная толщина лопасти;
    \item $c$ -- длина хорды лопасти.
\end{itemize}

На основе уравнения \ref{eq:propeler_model} упор движителя и момент сопротивления может быть записан следующим образом:
\begin{gather}
    T = \rho D^4K_T(J_0)n|n| \\
    Q = \rho D^5K_Q(J_0)n|n|
\end{gather}

\paragraph{Модель привода движителя}

Модель привода движителя обычно представляют следующим набором уравнений:
\begin{gather}
    L_a\frac{d}{dt}i_m=-R_ai_m-K_m n + V_m \\
    J_m\dot{n} = K_m i_m - Q
\end{gather}

\noindent где:
\begin{itemize}
    \item $V_m$ -- напряжение на обмотках привода (В);
    \item $i_m$ -- ток на обмотках привода (А);
    \item $n$ -- скорость вращения вала (об/с);
    \item $L_a, R_a$ -- индуктивность и сопротивление обмоток, соответственно;
    \item $K_m$ -- коэффициент крутящего момента мотора;
    \item $J_m$ -- момент инерции ротора.
\end{itemize}

Поскольку электрическая постоянная времени $T_a=L_a/R_a$ мала по сравнению с механической постоянной времени мотора, допустимо следующее предположение:
\begin{equation}
    \frac{L_a}{R_a}\frac{d}{dt}i_m \approx 0
\end{equation}
\noindent следовательно динамику движителя можно упростить до следующего состояния:
\begin{gather}
    \label{eq:motor_dynamics}
    0 = -R_a i_m-K_m w_m + V_m \\
    J_m\dot{n} = K_m i_m - Q
\end{gather}

При различных типах управления приводом, возможно представление его динамики в различной записи.

\paragraph{Управление током}
Пусть реализовано линейное пропорциональное управление током на обмотках мотора, тогда задаваемое напряжение на обмотках будет представлено следующим выражением:
\begin{equation}
    V_m = K_p(i_d - i_m),\: K_p>0
\end{equation}
\noindent где $i_d$ -- целевое значение тока, тогда уравнение \ref{eq:motor_dynamics} будет преобразовано следующим образом:
\begin{equation}
    (R_a + K_p)i_m = -K_mn + K_pi_d
\end{equation}

А модель динамики привода будет представлена в виде:
\begin{equation}
    \label{eq:motor_dynamic_current}
    J_m\dot{n} + \frac{K_m^2}{R_a+K_p}n = \frac{K_m K_p}{R_a + K_p}i_d - Q
\end{equation}

В случае если $K_p \gg R_a > 0$, то выражение можно упростить:
\begin{equation}
    J_m\dot{n} = K_m i_d - Q  
\end{equation}

\paragraph{Управление моментом}
Для мотором постоянного тока, создаваемый им момент будет пропорционален току на обмотках, таким образом целевое значение момента $Q_d$ может быть записано следующим образом:
\begin{equation}
    Q_d = K_m i_d
\end{equation}

При такой записи, из уравнения \ref{eq:motor_dynamic_current} уравнение динамики для данного режима управления может быть получено сразу же:
\begin{equation}
    J_m\dot{n} + \frac{K_m^2}{R_a+K_p}n = \frac{K_p}{R_a + K_p}Q_d - Q
\end{equation}

В случае если $K_p \gg R_a > 0$, то выражение упращается аналогичным образом:
\begin{equation}
    J_m\dot{n} = Q_d - Q  
\end{equation}

\paragraph{Управление напряжением на обмотках}
Уравнение динамики при таком режиме управления приводом может быть получено на базе уравнений \ref{eq:motor_dynamics} следующим образом:
\begin{equation}
    J_m\dot{n} + \frac{K_m^2}{R_a}n = \frac{K_m}{R_a}V_m - Q
\end{equation}

\subsection{Особенности работы движителей подруливающего типа}

\subsection{Модель рулевых устройств}
В техническом отчете \cite{steenson2011control} подробно представлен расчет подъемной силы для АНПА ``Dolphin 2'', в котором ДРК состоит из четырех кормовых рулей нормального расположения и четырех подруливающих движителей (2-х вертикальных и 2-х горизонтальных).

Подъемная сила формируемая горизонтальными рулевыми устройствами в стационарном случае определяется следующим выражением:
\begin{equation}
    F_{lift} = \frac{1}{2}\rho v^2 A_{SP} C_{LSP}
\end{equation}
\noindent где:
\begin{itemize}
    \item $\rho$ -- плотность воды;
    \item $v$ -- скорость продольного движения АНПА;
    \item $A_{SP}$ -- площадь РУ;
    \item $C_{LSP}$ -- подъемный коэффициент РУ;
\end{itemize}

В свою очередь, подъемный коэффициент РУ $C_{LSP}$ быть расcчитан следующим образом:
\begin{equation}
    C_{LSP} = a_{sp}\alpha \left( \frac{2 \pi}{1 + (2/AR)} \right)
\end{equation}
\noindent где:
\begin{itemize}
    \item $a_{sp}$ -- коэффициент подъемной силы (1/\degree);
    \item $\alpha$ -- угол поворота РУ относительно потока жидкости;
    \item $AR$ -- соотношение сторон РУ;
\end{itemize}

Соотношение сторон РУ $AR$ определяется следующим образом:
\begin{equation}
    AR = k \frac{S}{\bar{c}}
\end{equation}
\noindent где $S$ -- площадь РУ, $\bar{c}$ -- среднее значение длины РУ (chord), а коэффициент $k$ определяет влияние щели между РУ и корпусом ПА. Если щель отсутствует, то $k=2$, При отношении ширины щели к длине РУ около $0.1$ коэффициент $k\approx1.5$

Угол $\alpha$ определяет положение РУ относительно потока воды, в случае управления по глубине, можно эту переменную переписать следующим образом:
\begin{equation}
    \alpha = \delta - \theta
\end{equation}
\noindent где $\delta$ -- угол перекладки ГУ относительно ССК, $\theta$ -- текущий дифферент АНПА.

Традиционным конструктивным и эксплуатационным решением РУ является использование кормовых рулей и элеронов, хотя при малых скоростях движения находят применение и дополнительные рули глубины.
Управляющие моменты рулей являются функцией угла перекладки и скорости набегающего потока, при этом дополнительные силы лобового сопротивления не учитывается при анализе управления.
Модель РУ определяется следующими соотношениями \cite{боженов1986}:
\begin{gather}
    M_x^{rud} = m_x^{\delta} \cdot \delta_{\theta} \frac{\rho V^2}{2} U \\
    M_y^{rud} = m_y^{\delta} \cdot \delta_{\phi} \frac{\rho V^2}{2} U \\
    M_z^{rud} = m_z^{\delta} \cdot \delta_{H} \frac{\rho V^2}{2} U
\end{gather}
\noindent где:
\begin{itemize}
    \item $M_x^{rud}, M_y^{rud}, M_z^{rud}$ -- управляющие моменты РУ по крену, курсу и дифференту, соответственно;
    \item $m_x^{\delta}, m_y^{\delta}, m_z^{\delta}$ -- производные гидродинамических характеристик от перекладки рулей крена, направления и глубины, соответственно;
    \item $\delta_{\theta}, \delta_{\phi}, \delta_{H}$ -- углы перекладки рулей крена, направления и глубины, соответственно;
    \item $U$ -- водоизмещение аппарата, м$^3$.
\end{itemize}