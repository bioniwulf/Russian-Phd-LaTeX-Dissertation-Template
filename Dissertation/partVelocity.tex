\chapter{Разработка метода оценки скорости набегающего потока в устоявшемся режиме движения по изменению параметров работы маршевых движителей}\label{ch:Velocity}

Для корректного распределения управляющего воздействия критически важно понимание скорости набегающего потока, так как это существенным образом влияет (как это было показано в главе \ref{sec:Propulsion/Models}) на силы и моменты которые формируются исполнительными механизмами движительно-рулевого комплекса.

\section{Анализ модели гребного винта}
Упор создаваемый гребным винтом движителя $T$ и момент сопротивления вращению $Q$ формируемый при этом на валу определяется следующими формулами \cite{пантов1973основы}:
\begin{gather}
    T(n, J_0) = K_T (J_0) \rho D^4 n |n| \\
    Q(n, J_0) = K_Q (J_0) \rho D^5 n |n|
\end{gather}
\noindent где:
\begin{itemize}
    \item $K_T (J_0), K_Q (J_0)$ -- соответственно, коэффициенты упора и момента;
    \item $D$ -- диаметр винта;
    \item $\rho$ -- плотность воды;
    \item $n$ -- скорость вращения гребного винта;
\end{itemize}

Переменная $J_0$ представляет собой относительную поступь и определяется следующим уравнением:
\begin{equation}
    J_0 = \frac{u_a}{nD}
\end{equation}
\noindent где $u_a$ -- скорость набегающего потока воды относительно ПА.
Эта скорость не совсем равна скорости аппарата $u$ и в устоявшемся режиме связана с ней следующим соотношением \cite{lewis1988principles}:
\begin{equation}
    u_a = (1-w)u
\end{equation}
\noindent где $w$ -- коэффициент попутного потока.
Параметр $w$ находится в диапазоне $0.1-0.4$ в зависимости от конструктивных особенностей движителя и ПА.

Более подробно параметр $w$ представлен в работе \cite{10.1016/s1474-6670-17-46514-1} которая опирается лекционные заметки \cite{walderhaug1992motstand}:

\begin{equation}
    \label{eq:ambient_flow}
    u_a = (1-w)u = (1 - w_w - w_p - w_v)u
\end{equation}
\noindent где:
\begin{itemize}
    \item $w_p$ -- составная часть коэффициента попутного потока отражающая влияние так называемых потенциальных эффектов связанных с влиянием движения корпуса судна или ПА на скорость движения воды;
    \item $w_v$ -- составная часть коэффициента попутного потока отражающая влияние вязких эффектов из-за влияния пограничных слоев.
\end{itemize}

Коэффициенты упора и момента гребного винта $K_T (J_0), K_Q (J_0)$ в больших пределах могут быть линейно аппроксимированы следующим образом \cite{10.1109/48.838987}:
\begin{gather}
    \label{eq:coef_1}
    K_T(J_0) = \alpha_1 - \alpha_2 J_0 \\
    K_Q(J_0) = \beta_1 - \beta_2 J_0
\end{gather}
\noindent где $\alpha_i, \beta_i$ -- безразмерные положительные константы.

Такая линейная аппроксимация позволяет утверждать что коэффициенты $\alpha_1, \beta_1$ определяют эффективность гребного винта при отсутствии набегающего потока.

Таким образом уравнения \ref{eq:coef_1} могут быть переписаны в следующей форме:
\begin{gather}
    K_T = K_T^{bp} - \alpha_1 J_0 \\
    K_Q = K_Q^{bp} - \beta_1 J_0
\end{gather}
\noindent где $K_T^{bp}, K_Q^{bp}$ -- коэффициенты упора и момента гребного винта полученные в результате швартовых испытаний движителя.