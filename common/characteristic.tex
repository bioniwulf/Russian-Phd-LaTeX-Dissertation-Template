{\actuality}
Задача управления ДРК (control allocation в иностранной терминологии) возникает естественным образом для избыточного (over-actuated) ДРК ПА, в котором ИМ (рули управления, маршевые и подруливающие движители) больше чем количество доступных для управления степеней свободы. Использование ДРК такого типа широко распространено по следующим причинам:

\begin{itemize}
    \item в виду того, что подводные операции сопряжены с высокой степенью риска, избыточность необходима для реализации функций резервирования систем управления движением подводного аппарата;
    \item в силу гидродинамических особенностей функционирования при различных режимах движения (позиционный, крейсерский) часть исполнительных механизмов ДРК будет работать лучше или хуже; в соответствии с этим, при разработке многоцелевых подводных аппаратов необходимо использовать избыточные комплекты ИМ \cite{valasek2002design}.
    \item избыточная конфигурация ДРК при энергетически оптимальном управлении ДРК позволяет сократить затраты энергии на 20-25\% по сравнению с эквивалентной неизбыточной конфигурацией \cite{бриллиантов2005разработка}.
\end{itemize}

Решение задачи управления ДРК позволяет оптимизировать энергетические затраты на движение ПА, обеспечить отказоустойчивость системы управления, и уменьшить механический износ ИМ в условиях накладываемых на них ограничений \cite{enns1998control}.
Исторически первой эта задача возникла в двух областях: для систем управления многостепенными манипуляторами \cite{craig2009introduction} и самолётами \cite{bordignon1996constrained}. 
Основной целью тогда было создание систем аккомодации, то есть формирования такой системы управления движением, которая была бы устойчива к выходу из строя отдельных исполнительных механизмов.

В общем случае задача управления ДРК для случая избыточных систем управления ведет к задаче численной оптимизации с линейными ограничениями, решение которой сложно реализовать для управляющих контуров в условиях операционных систем реального времени в виду особенностей итерационного поиска многих методов решения оптимизационных задач \cite{burken2001two}.

{\progress}
В настоящее время имеется достаточное количество обзорных статей в иностранной научной литературе, посвященных задаче управления ДРК, включая как отдельные приложения – суда и подводные аппараты \cite{fossen2006survey}, летательные средства \cite{oppenheimer2006control}, так и междисциплинарные \cite{10.1016/j.automatica.2013.01.035} в рамках которой рассматриваются различные методы управления ДРК в приложении к различным областям робототехники.
В отечественной литературе такая постановка задачи обычно не выделяется в отдельный класс \cite{филаретов2016особенности} и, например, задача аккомодации к отказам исполнительных механизмов решается в рамках управления объектом целиком \cite{мартынова2020подход, зыбин2014аналитическое}.
Впрочем, упоминание задачи управления ДРК можно найти в отдельных статьях \cite{амбросовский2013распределение, воловодов2003распределение, амбросовский2014алгоритмы} и главах диссертаций \cite{власов2016адаптивное}.

Актуальность задачи управления ДРК описана в трудах отечественных ученых Армишева С.В. \cite{армишев86} Воловодова С.К. \cite{воловодов2003распределение,воловодов2005особенности} Амбрасовского В.М. \cite{амбросовский2013распределение}, а также в трудах зарубежных ученых: Wayne C. Durham \cite{durham1993constrained, bordingnon1995closed, durham1994constrained, durham2001computationally} Ola Härkegård \cite{10.1016/j.automatica.2004.09.007, harkegaard2003backstepping, harkegaard2004dynamic}, коллектива учёных Thor I. Fossen и Tor A. Johansen (\cite{fossen2006survey, 10.1016/j.automatica.2013.01.035, 10.1109/tcst.2003.821952}), Alessandro Baldini \cite{baldini2018constrained, baldini2020constrained}, Edin Omerdic \cite{ahmad2007control, omerdic2020geometric, capocci2018fault}, Mehdi Naderi \cite{naderi2019guaranteed, naderi2019fault} и др.

{\object} является движительно-рулевой комплекс автономного необитаемого подводного аппарата.

{\subject} является алгоритмы и методы управления ДРК автономного необитаемого подводного аппарата, методы идентификации параметров его работы, численные методы и комплексы программ для реализации методов управления им.
% \ifsynopsis
% Этот абзац появляется только в~автореферате.
% Для формирования блоков, которые будут обрабатываться только в~автореферате,
% заведена проверка условия \verb!\!\verb!ifsynopsis!.
% Значение условия задаётся в~основном файле документа (\verb!synopsis.tex! для
% автореферата).
% \else
% Этот абзац появляется только в~диссертации.
% Через проверку условия \verb!\!\verb!ifsynopsis!, задаваемого в~основном файле
% документа (\verb!dissertation.tex! для диссертации), можно сделать новую
% команду, обеспечивающую появление цитаты в~диссертации, но~не~в~автореферате.
% \fi

{\aim} диссертационной работы является разработка методов энергоэффективного и отказоустойчивого управления избыточным движительно-рулевым комплексом подводного аппарата с учётом динамики исполнительных механизмов и их гидродинамических особенностей.

Для~достижения поставленной цели необходимо было решить следующие {\tasks}:
\begin{enumerate}[beginpenalty=10000] % https://tex.stackexchange.com/a/476052/104425
  \item Изучить модели описания исполнительных механизмов (маршевые движители, подруливающие движители, рули управления) ДРК подводного аппарата учитывающие особенности их работы в набегающем потоке;
  \item Разработать и реализовать метод оценки скорости движения ПА относительно набегающего потока по параметрам работы электроприводов маршевых движителей;
  \item Разработать и реализовать энергетически оптимальный метод управления ДРК для АНПА адаптивный к различным типам движения ПА;
  \item Провести проверку эффективности разработанного метода управления ДРК на базе модельных экспериментов.
\end{enumerate}

{\novelty}
\begin{enumerate}[beginpenalty=10000] % https://tex.stackexchange.com/a/476052/104425
  \item Разработан метод оценки скорости движения ПА относительно набегающего потока по изменению параметров работы маршевых движителей в установившемся режиме движения.
  В результате обработки данных запусков АНПА ``ММТ-3000'' расхождение скорости оцененной по предлагаемому методу с эталонными показаниями доплеровского лага составило не более чем 0.18 м/с на переходных режимах и не более чем 0.1 м/с на установившихся режимах движения.
  \item Предложен новый метод энергетический оптимального управления ДРК ПА, который учитывает особенности работы исполнительных механизмов в набегающем потоке и способен адаптивно перераспределять управляющие воздействия при вариациях скорости.
  По сравнению с используемым в АНПА ``ММТ-300'', предлагаемый метод позволяет обеспечивать более широкий диапазон управления и позволяет удерживать заданный момент при более высоких значениях целевой скорости.
\end{enumerate}
% \begin{noteplan}
%     В новизне надо будет показать конкретные числовые преимущества каждого из предложений по сравнению с существующими вариантами. Уже по результатам моделироваия и испытаний будет на чистовую написано.
% \end{noteplan}
{\influence} разработанных методов заключается в надежном определении скорости движения АНПА в набегающем потоке и в обеспечении непрерывного перераспределения управляющих команд ДРК при различных вариациях скорости движения подводного аппарата.

{\methods} В диссертационной работе использовались методы теории управления, линейной и нелинейной оптимизации; методы математического и имитационного моделирования, современные средства разработки программных комплексов и моделирования.

{\defpositions}
\begin{enumerate}[beginpenalty=10000] % https://tex.stackexchange.com/a/476052/104425
  \item Метод оценки скорости движения ПА относительно набегающего потока по изменению параметров работы маршевых движителей в установившемся режиме движения;
  \item Метод энергетический оптимального управления ДРК ПА, который учитывает особенности работы исполнительных механизмов в набегающем потоке и способен адаптивно перераспределять управляющие воздействия при вариациях скорости;
  \item Программная реализация энергетически-оптимального управления ДРК адаптивного к вариациям скорости.
\end{enumerate}

{\reliability} Достоверность полученных выводов определяется корректным применением методов математического моделирования сложных систем и обработки экспериментальных данных. Обоснованность полученных методов и алгоритмов основывается на сопоставлении полученных результатов с результатами, полученными другими известными методами и алгоритмами.

{\probation}
Основные результаты диссертации
докладывались и обсуждались на следующих конференциях:
\begin{enumerate}
    \item ``Управление в морских и аэрокосмических системах (УМАС-2016)'' (Санкт-Петербург, 04–06 октября 2016 года);
    \item Всероссийская научно-техническая конференция ``Технические проблемы освоения мирового океана'' (г. Владивосток, 2017);
    \item Всероссийская научно-техническая конференция ``Технические проблемы освоения мирового океана'' (г. Владивосток, 30 сентября – 3 октября 2019 г.);
    \item ``2019 IEEE International Underwater Technology Symposium, UT 2019'' (Kaohsiung, 16–19 апреля 2019 года);
    \item Международная конференция по промышленному инжинирингу и современным технологиям ``FarEastCon-2019'' (г. Владивосток, 1-4 сентября 2019 года);
    \item Международная конференция по промышленному инжинирингу и современным технологиям ``FarEastCon-2020'' (г. Владивосток, 5-8 октября 2019 года);
    \item Конференци ``Управление в морских системах'' (УМС-2000) в рамках 13-й мультиконференции по проблемам управления (МКПУ-2020) (г. Санкт-Петербург, 6-8 Октября 2020 г.).
\end{enumerate}

{\contribution} Автор принимал участие в постановке цели и задач по теме исследования, разработал и реализовал программно методы и алгоритмы, предложенные в данной работе, обработал и проанализировал экспериментальные данные, участвовал в обсуждении полученных результатов, подготовке научных статей, материалов конференций, участвовал на конференциях и семинарах.

\ifnumequal{\value{bibliosel}}{0}
{%%% Встроенная реализация с загрузкой файла через движок bibtex8. (При желании, внутри можно использовать обычные ссылки, наподобие `\cite{vakbib1,vakbib2}`).
    {\publications} Основные результаты по теме ВКР изложены
    в~XX~печатных изданиях,
    X из которых изданы в журналах, рекомендованных ВАК,
    X "--- в тезисах докладов.
}%
{%%% Реализация пакетом biblatex через движок biber
    \begin{refsection}[bl-author, bl-registered]
        % Это refsection=1.
        % Процитированные здесь работы:
        %  * подсчитываются, для автоматического составления фразы "Основные результаты ..."
        %  * попадают в авторскую библиографию, при usefootcite==0 и стиле `\insertbiblioauthor` или `\insertbiblioauthorgrouped`
        %  * нумеруются там в зависимости от порядка команд `\printbibliography` в этом разделе.
        %  * при использовании `\insertbiblioauthorgrouped`, порядок команд `\printbibliography` в нём должен быть тем же (см. biblio/biblatex.tex)
        %
        % Невидимый библиографический список для подсчёта количества публикаций:
        \printbibliography[heading=nobibheading, section=1, env=countauthorvak,          keyword=biblioauthorvak]%
        \printbibliography[heading=nobibheading, section=1, env=countauthorwos,          keyword=biblioauthorwos]%
        \printbibliography[heading=nobibheading, section=1, env=countauthorscopus,       keyword=biblioauthorscopus]%
        \printbibliography[heading=nobibheading, section=1, env=countauthorconf,         keyword=biblioauthorconf]%
        \printbibliography[heading=nobibheading, section=1, env=countauthorother,        keyword=biblioauthorother]%
        \printbibliography[heading=nobibheading, section=1, env=countregistered,         keyword=biblioregistered]%
        \printbibliography[heading=nobibheading, section=1, env=countauthorpatent,       keyword=biblioauthorpatent]%
        \printbibliography[heading=nobibheading, section=1, env=countauthorprogram,      keyword=biblioauthorprogram]%
        \printbibliography[heading=nobibheading, section=1, env=countauthor,             keyword=biblioauthor]%
        \printbibliography[heading=nobibheading, section=1, env=countauthorvakscopuswos, filter=vakscopuswos]%
        \printbibliography[heading=nobibheading, section=1, env=countauthorscopuswos,    filter=scopuswos]%
        %
        \nocite{*}%
        %
        {\publications} Основные результаты по теме ВКР изложены в~\arabic{citeauthor}~печатных изданиях,
        \arabic{citeauthorvak} из которых изданы в журналах, рекомендованных ВАК\sloppy%
        \ifnum \value{citeauthorscopuswos}>0%
            , \arabic{citeauthorscopuswos} "--- в~периодических научных журналах, индексируемых Web of~Science и Scopus\sloppy%
        \fi%
        \ifnum \value{citeauthorconf}>0%
            , \arabic{citeauthorconf} "--- в~тезисах докладов.
        \else%
            .
        \fi%
        \ifnum \value{citeregistered}=1%
            \ifnum \value{citeauthorpatent}=1%
                Зарегистрирован \arabic{citeauthorpatent} патент.
            \fi%
            \ifnum \value{citeauthorprogram}=1%
                Зарегистрирована \arabic{citeauthorprogram} программа для ЭВМ.
            \fi%
        \fi%
        \ifnum \value{citeregistered}>1%
            Зарегистрированы\ %
            \ifnum \value{citeauthorpatent}>0%
            \formbytotal{citeauthorpatent}{патент}{}{а}{}\sloppy%
            \ifnum \value{citeauthorprogram}=0 . \else \ и~\fi%
            \fi%
            \ifnum \value{citeauthorprogram}>0%
            \formbytotal{citeauthorprogram}{программ}{а}{ы}{} для ЭВМ.
            \fi%
        \fi%
        % К публикациям, в которых излагаются основные научные результаты диссертации на сометоды иискание учёной
        % степени, в рецензируемых изданиях приравниваются патенты на изобретения, патенты (свидетельства) на
        % полезную модель, патенты на промышленный образец, патенты на селекционные достижения, свидетельства
        % на программу для электронных вычислительных машин, базу данных, топологию интегральных микросхем,
        % зарегистрированные в установленном порядке.(в ред. Постановления Правительства РФ от 21.04.2016 N 335)
    \end{refsection}%
    \begin{refsection}[bl-author, bl-registered]
        % Это refsection=2.
        % Процитированные здесь работы:
        %  * попадают в авторскую библиографию, при usefootcite==0 и стиле `\insertbiblioauthorimportant`.
        %  * ни на что не влияют в противном случае
        % \nocite{vakbib2}%vak
        % \nocite{patbib1}%patent
        % \nocite{progbib1}%program
        % \nocite{bib1}%other
        % \nocite{confbib1}%conf
    \end{refsection}%
        %
        % Всё, что вне этих двух refsection, это refsection=0,
        %  * для диссертации - это нормальные ссылки, попадающие в обычную библиографию
        %  * для автореферата:
        %     * при usefootcite==0, ссылка корректно сработает только для источника из `external.bib`. Для своих работ --- напечатает "[0]" (и даже Warning не вылезет).
        %     * при usefootcite==1, ссылка сработает нормально. В авторской библиографии будут только процитированные в refsection=0 работы.
}

% При использовании пакета \verb!biblatex! будут подсчитаны все работы, добавленные
% в файл \verb!biblio/author.bib!. Для правильного подсчёта работ в~различных
% системах цитирования требуется использовать поля:
% \begin{itemize}
%         \item \texttt{authorvak} если публикация индексирована ВАК,
%         \item \texttt{authorscopus} если публикация индексирована Scopus,
%         \item \texttt{authorwos} если публикация индексирована Web of Science,
%         \item \texttt{authorconf} для докладов конференций,
%         \item \texttt{authorpatent} для патентов,
%         \item \texttt{authorprogram} для зарегистрированных программ для ЭВМ,
%         \item \texttt{authorother} для других публикаций.
% \end{itemize}
% Для подсчёта используются счётчики:
% \begin{itemize}
%         \item \texttt{citeauthorvak} для работ, индексируемых ВАК,
%         \item \texttt{citeauthorscopus} для работ, индексируемых Scopus,
%         \item \texttt{citeauthorwos} для работ, индексируемых Web of Science,
%         \item \texttt{citeauthorvakscopuswos} для работ, индексируемых одной из трёх баз,
%         \item \texttt{citeauthorscopuswos} для работ, индексируемых Scopus или Web of~Science,
%         \item \texttt{citeauthorconf} для докладов на конференциях,
%         \item \texttt{citeauthorother} для остальных работ,
%         \item \texttt{citeauthorpatent} для патентов,
%         \item \texttt{citeauthorprogram} для зарегистрированных программ для ЭВМ,
%         \item \texttt{citeauthor} для суммарного количества работ.
% \end{itemize}
% % Счётчик \texttt{citeexternal} используется для подсчёта процитированных публикаций;
% % \texttt{citeregistered} "--- для подсчёта суммарного количества патентов и программ для ЭВМ.

% Для добавления в список публикаций автора работ, которые не были процитированы в
% автореферате, требуется их~перечислить с использованием команды \verb!\nocite! в
% \verb!Synopsis/content.tex!.
