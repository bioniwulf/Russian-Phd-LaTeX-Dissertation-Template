%% Согласно ГОСТ Р 7.0.11-2011:
%% 5.3.3 В заключении диссертации излагают итоги выполненного исследования, рекомендации, перспективы дальнейшей разработки темы.
%% 9.2.3 В заключении автореферата диссертации излагают итоги данного исследования, рекомендации и перспективы дальнейшей разработки темы.
\begin{enumerate}
  \item Предложен новый метод оценки скорости движения АНПА относительно набегающего потока по изменению параметров работы движителей маршевой группы в установившемся режиме движения. Метод позволяет оценить скорость движения АНПА относительно потока в случае отсутствия основного датчика скорости или выхода его из строя.
  \item Проведена верификация предолженного метода оценки скорости набегающего на обработке данных запусков АНПА ``ММТ-3000''. После уточнения параметров ГВ на калибровочной части маршрута, расхождение оцениваемой скорости с эталонными показаниями доплеровского лага составило не более чем 0.18 м/с на переходных режимах и не более чем 0.1 м/с на установившихся режимах движения со среднеквадратичным отклонением $\sigma=0.17$ м/с.
  \item Предложен новый алгоритм управления ДРК подводного аппарата, который учитывает особенности работы исполнительных механизмов в набегающем потоке и способен адаптивно перераспределять управляющие воздействия при вариациях скорости.
  \item Показано что предложеный метод управления ДРК обеспечивает более высокую скорость работы по сравнению со стандартными подходами к решению задачи управления ДРК через формирование линейной и квадратичной оптимальной задачи при сопосповимой области допустимых значений. По сравнению с используемым в АНПА ``ММТ-300'', предлагаемый метод позволяет обеспечивать более широкий диапазон управления и позволяет удерживать заданный момент при более высоких значениях целевой скорости.
\end{enumerate}
